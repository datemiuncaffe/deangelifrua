\chapter[]{RIFERIMENTI}

Dal periodico della Snia Viscosa 1938-1942. Numeri vari
RIFERIMENTI


Ideatore dell’opera: 	                          	  Loredano Tavazzi
Elaborazione immagini e inserimento testi:
     Giancarlo Soave
            Niccolò Zucchi Frua
Impostazione, coordinamento, ricerca, 
scelta immagini e didascalie:                          Loredano Tavazzi                                     



I documenti per i fazzoletti di istruzione militare provengono dall’archivio del Dott. Dirk Ziesing

Tutti gli altri documenti provengono dall’archivio 
 Loredano Tavazzi



Per le traduzioni da tedesco, francese e inglese hanno contribuito:

 Alessandro Porro, 		docente dell’Università di Brescia
Cinzia Pozzi. 						     interprete
Carlo Solarino , 					    giornalista 	







AUTORI DEI TESTI

- Marina Frua -
Premessa


-Dirk Ziesing- 
Military instruction hanndkerchief of the British Empire
Italien- De Angeli Frua


-Loredano Tavazzi -
Estrapolazioni dai testi originali di Dirk Ziesing e di alcuni significativi particolari dai singoli manufatti.
Note esplicative su fazzoletti cavalleria e armi pesanti.
Note esplicative su fazzoletti 1915–1918.
Nota esplicativa sul foulard Etiopia.
Nota esplicativa sui tessuti artificiali.
Note esplicative su pubblicità e promozione.



INDICE DEI TESTI E ARGOMENTI

Premessa 						         pag.   5

Capitolo I
La nascita dei fazzoletti  per  istruzione militare          pag.    7                                  
Military instruction handkerchief
		of the British Empire  				       pag.      8
I fazzoletti con istruzioni militari dell’Impero Britannico  							                    pag.     9                                                        
Il primo fazzoletto britannico con istruzioni                 pag.  11
Il secondo  fazzoletto britannico con istruzioni             pag.  17
Il terzo  fazzoletto britannico con istruzioni                   pag. 25
L’origine dei fazzoletti con istruzioni militari               pag. 31


Capitolo II
I fazzoletti per istruzione militare in Italia 
prima parte: descrizione storica                                  pag.  37                                                        
  De Angeli Frua                                      		         pag.  39
Sguardo d’insieme     			  	          pag. 39
Il fazzoletto italiano n°1	                   		          pag. 39
Il fazzoletto italiano n°2                           		          pag. 43
Il fazzoletto italiano n°3	                      		          pag. 43
Il fazzoletto italiano con carta geografica del 1912         pag. 59
Il fazzoletto italiano con carta geografica del 1918         pag. 61
I rapporti tra Francia e Italia	                    		          pag. 61

Capitolo III
I fazzoletti per istruzione militare in Italia - Seconda parte                                                                                           pag. 65                                                                            pag.   65                           
De Angeli Frua			               	          pag. 66
Sguardo d’insieme			         		         pag.  67
Il fazzoletto italiano n° 1                              	         pag.  68 
Il fazzoletto italiano n° 2                               	         pag.  68
Il fazzoletto italiano n° 3                              	         pag.  69
Fazzoletto cartografico italiano del 1912      	         pag.  70
Fazzoletto cartografico italiano del 1918       	          pag. 72
Nota esplicativa                                             	         pag.  72
Altre produzioni dell’epoca                          	         pag.  95
Specifica su fazzoletto Etiopia 1935-36 		         pag.  97

Capitolo IV 
Tessuti disegni e trame del periodo 1930-1943               pag. 99
Capitolo V
Tessuti artificiali del periodo 1938-1943             	        pag. 109
Capitolo VI
Tovaglie, tovaglioli e altro del periodo 1950-1965        pag. 119    
Capitolo VII
Strofinacci da cucina  del periodo  1965-1970               pag. 127                                             
Capitolo VIII
Pubblicità, promozioni, immagine 		 	        pag. 137
Specifica sulla pubblicità  				       pag.  138

Riferimenti		           			        pag. 155
Autori dei testi				                    pag. 156
INDICE DELLE ILLUSTRAZIONI

Ciclo di lavorazione della seta, anno 1933  	          pag.   2
Prima pagina della rivista inglese sulla 
nascita dei fazzoletti di istruzione militare   	          pag. 12                              
Fazzoletto militare britannico n°1		                     pag.  13
Particolari del fazzoletto n°1   		                pag. 14-15
Fazzoletto militare britannico n°2 		                     pag.  18
Particolare del fazzoletto n°2  		                     pag.  19
Fazzoletto militare britannico della marina n°3           pag. 22
Fazzoletto britannico della Marina -particolare 	          pag. 23
Fazzoletto militare italiano n°1                                    pag.   40
Fazzoletto militare italiano n°2                                     pag.  41
Particolare del fazzoletto militare italiano n°2A         pag.  44
Fazzoletto militare italiano n°3                                     pag.  45      
Fazzoletto militare italiano n°4 –Cavalleria-     	          pag. 52
Fazzoletto delle armi pesanti                   		          pag. 53
Fazzoletto militare italiano n°5 cartografico 1912       pag. 56
Fazzoletto militare italiano n°6 cartografico 1918       pag. 57
Dettaglio in basso a sinistra del fazzoletto n°1              pag. 62
Documento descrizione contenuti fazzoletto n°1          pag. 63
Richiesta di G. Frua per stampa fazzoletti militari      pag. 64
Certificazione della produzione De Angeli                     pag. 73
Particolari del fazzoletto militare italiano n°2   pag. 73-75-76
Particolare del fazzoletto militare italiano n°1            pag.  74
Particolari del fazzoletto militare italiano n°2A    pag.77 a 81
Fazzoletto militare italiano n°4 per Cavalleria             pag. 80
Particolari del fazzoletto militare italiano n°4           pag. 81-83          
Particolare del fazzoletto militare italiano n°3   pag. da 82 a 84     
Particolari del fazzoletto militare italiano n°4         pag. 88-91        
Cartoline della Regia Cavalleria 1920-1930              pag. 92-93
Particolare del fazzoletto militare italiano n°5            pag.  94    
Fazzoletto di educazione famigliare                              pag.  95
Fazzoletto di educazione civica                                     pag.  96
Foulard  della guerra d’Etiopia 1935-1936                   pag.  97
Produzione tessile di fine ‘800                                        pag.  98
Disegni di tessuti                                                     pag. 100-108
“I tessili nuovi” 1941, copertina                                   pag. 110
“I tessili nuovi” 1941, pagine interne                     pag. 112-114
Logo Snia Viscosa                                                          pag. 114
La stilista Biki (da Corsera)                                          pag. 115
Ciclo di produzione del nylon                                        pag. 116
Ciclo di produzione del nàion e del fiocco                     pag. 117
Particolari tovaglie                                                 pag. 120-122
Prove produzione tovaglie e tovaglioli                         pag. 123
Tovaglietta per prima colazione                                   pag. 124
Grembiule da cucina                                                      pag. 125
Strofinacci da cucina                                             pag. 128-136
Intestazione “Mani di fata”                                          pag. 139
Intestazione “L’illustrazione italiana”                         pag. 139
“Mani di fata”, pagine interne                               pag. 140-142
Locandina tramviaria                                                    pag. 143
Fazzoletti con personaggi W. Disney                           pag. 144
Cartoline                                                                  pag. 144-148
Pagine e immagini pubblicitarie                            pag. 149-154


\clearpage
