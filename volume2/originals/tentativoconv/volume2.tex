\begin{verbatim}
                           De Angeli Frua

                        Un secolo di tessuti









                                [pic]

             Ciclo di lavorazione della seta, anno 1933

                           De Angeli Frua
                           Secondo volume
                        Un secolo di tessuti
                          dal 1872 al 1970




                                [pic]

                                [pic]





                         © Loredano Tavazzi
\end{verbatim}

SI RINGRAZIANO PER AVERE OFFERTO NOTIZIE E DOCUMENTAZIONI:

Il dott. Dirk Ziezing per i testi in inglese e tedesco dedicati ai
fazzoletti di istruzione militare e relativi documenti, usati dagli
eserciti britannico e italiano.

Tutte le persone che hanno fornito informazioni e materiali utili alla
compilazione del volume.

Gli estensori dei testi si scusano per eventuali imprecisioni anche
nelle didascalie delle immagini.

L' editore, pur avendo fatto il possibile per evitarli, si scusa per
possibili errori od omissioni nella citazione delle fonti ed è a
disposizione degli aventi diritto.

\begin{verbatim}
                              Premessa
\end{verbatim}

di Marina Frua

``Memorie della De Angeli Frua che meritano di essere conosciute e
trasmesse''

\begin{verbatim}
A Milano in via Paleocapa al numero  uno,  una  bella  casa  di  quattro
\end{verbatim}

piani vicino alla Stazione Nord e al Parco Sempione, abitavano i De
Angeli e i Frua. Ernesto De Angeli, senatore del Regno, e le sue belle,
severe e impettite sorelle occupavano il primo e il secondo piano.
Giuseppe Frua abitava al terzo piano e Alberto, figlio di Giuseppe, al
quarto. Io facevo parte della famiglia di Alberto: ero piccola e di
questi personaggi, all'epoca così importanti, ho un ricordo forse un po'
sbiadito ma a tratti intenso. Sovente andavo a trovare mio nonno
Giuseppe. Lui mi dava le arance affettate con lo zucchero e le violette
candite e controllava che il mio vestitino fosse tassativamente di
cotone: anche d'inverno. Ricordo il suo sorriso: con un lampo d'intesa i
suoi occhi si accendevano di una luce speciale. Portava dei piccoli
occhiali con una semplice montatura di metallo, per correggere il suo
astigmatismo che qualcuno di noi ha ereditato. Il nonno Giuseppe era un
grande lavoratore e ha trasmesso a tutti i suoi collaboratori la
passione per il lavoro, l'amore per la famiglia, il rispetto per gli
uomini e le cose, l'importanza dell'istruzione, dello studio e della
religione. Aveva passato l'intera vita a creare con costanza e tenacia
prodotti tessili di alta qualità in un'epoca in cui, all'estero, si
diceva che l'Italia sapeva produrre solo arance e mandarini. Lui invece
riuscì a far conoscere e apprezzare in tutta Europa i suoi tessuti fra i
quali la famosa ``Costella'', una stoffa dai piccoli disegni con il
marchio ``Sole e Onda''. Si occupava dei suoi collaboratori e operai
come un padre e per loro creò abitazioni, asili e centri di assistenza
sanitaria. Solo più tardi, crescendo, ho appreso quale impegnativo e
prezioso lavoro ha fatto e quante drammatiche cose sono poi successe
nella seconda guerra mondiale e nella difficile ripresa, con la
distruzione e la chiusura delle fabbriche. Tutto disperso: filature,
tessiture, stamperie, tessuti e storia. Si era dissolta anche la
memoria. Ma, nel quartiere Frua, le case di chi lavorava in fabbrica
esistono ancora. E proprio in via Moncalvo abita e opera una persona che
con tenacia ha voluto che il ricordo di tutto quel lavoro non andasse
perduto: Loredano Tavazzi. Un personaggio eccezionale che ha passato una
vita in quella via Moncalvo e che, ripensando ai valori creati e
trasmessi da Giuseppe Frua, con passione intelligenza costanza e forse
anche un pizzico di meritata fortuna, ha raccolto testimonianze e ha
conservato ricostruito e descritto documenti che, altrimenti, sarebbero
andati perduti. In questa sua ostinata ricerca ha recuperato, della De
Angeli Frua, marchi, tessuti, fotografie, cartoline e manifesti
pubblicitari. Ne è nato così questo volume -- a seguito del precedente
``De Angeli Frua, una famiglia, un'industria nella storia di Milano'' --
che evidenzia, tra le tante altre cose, la particolare scoperta, a molti
credo sconosciuta, dei ``fazzoletti militari'' che rievocano momenti di
storia europea pieni di guerre ma anche di analfabetismo. Forniti in
dotazione ai soldati, questi fazzoletti illustravano con stampe e
disegni attenti e dettagliati, sovente ricchi di vere soluzioni
artistiche, come usare le armi ma poi come montare a cavallo o come
disporsi sul campo. A Loredano Tavazzi va dunque il mio sentito e
affettuoso ringraziamento e quello dei miei figli e nipoti che portano,
oltre al proprio, il cognome Frua, per questo straordinario lavoro sulla
conservazione di memorie della De Angeli Frua, che meritano di essere
conosciute e trasmesse.

Milano 2014

{[}pic{]}

Capitolo I

\begin{verbatim}
                             La nascita
                           dei fazzoletti
                                 per
                         istruzione militare








                          di Dirk Ziesling
\end{verbatim}

{[}pic{]}

\begin{verbatim}
I FAZZOLETTI CON ISTRUZIONI MILITARI
DELL’IMPERO BRITANNICO
\end{verbatim}

Quando il 19° secolo arrivò alla sua fine, strategie di guerra, tattiche
e armamenti militari videro significativi cambiamenti. Questo fu il
periodo della creazione dei fazzoletti illustrati, allo scopo di fornire
istruzioni militari e utili informazioni per il servizio militare e
navale. Un ufficiale della fanteria dell'esercito Britannico ebbe l'idea
di queste istruzioni militari illustrate su fazzoletti. Il suo nome è
Carrè Fulton, nato il 15 marzo 1848 a Douglas, nell'Isola di Man, figlio
gemello del Tenente-Colonnello William Fulton. Egli iniziò la sua
carriera militare nel 1867 come fante del 15° Reggimento a piedi.
Seguirono 3 anni di servizio oltremare in Giamaica e nelle Bermuda. Nel
1871 egli tornò in Gran Bretagna e fu promosso al grado di tenente e un
anno dopo entrò a far parte del 68° Reggimento a piedi. Nel 1877 divenne
capitano e nel 1883 maggiore, questa volta della fanteria leggera di
Durham. Durante gli anni 1885 e 1886 egli servì il corpo militare di
frontiera di stanza in Sudan e prese parte alla campagna contro le
truppe della resistenza anticoloniale del Mahdi. 1 (vedi nota a pag.
36). Per la sua partecipazione alla battaglia di Ginnis, il 30 dicembre
1885 egli ricevette la medaglia per la campagna di Egitto insieme alla
stella di Khedive. Fu dopo questa battaglia che le truppe Britanniche
abbandonarono il classico colore rosso delle loro uniformi a favore del
colore cachi, quale principio di una nuova era. Nel 1887, Fulton si
ritirò con la carica di sottotenente-colonnello della fanteria leggera
di Durham. Successivamente egli lavorò come poliziotto a Gibilterra. Al
cambio del secolo, si sa che egli viveva nel quartiere di Acton a
Londra, poiché il 16 novembre 1901 il Times menzionò il colonello Carrè
Fulton in quanto vittima di una banda di ladri. Egli morì nel 1911 a
Guernsey, un'isola del Canale della Manica. Suo figlio ventenne, il
sottotenente Cecil John Fulton fu ucciso nell'aprile del 1916 e fu
seppellito nel cimitero della città di Bethune (Francia).Carrè Fulton
era un discendente di una famiglia di fabbricanti di stoffa. Nel 18°
secolo il suo antenato Humphrey Fulton stabilì la manifattura della
garza di seta in Scozia. {[}pic{]}

I figli di Humphrey, William, Henry e Robert di Lochliboside e Hartfield
continuarono l'attività e il 19° secolo vide l'azienda William Fulton \&
figli Ltd nella città di Paisley (Scozia). All'inizio essi si definirono
produttori di garza e mussola; la mussola, un cotone intessuto
finemente, divenne la stoffa utilizzata per i fazzoletti con le
istruzioni. Gli scozzesi Fulton mantennero l'attività fino al 1985
quando vennero acquisiti da un'altra società tessile. Malgrado nessuno
dei citati fazzoletti con le istruzioni mostri alcuna indicazione del
fabbricante, è molto probabile che Carrè li stampò nell'azienda di
famiglia.

IL PRIMO FAZZOLETTO BRITANNICO CON ISTRUZIONI

Carrè Fulton ottenne 3 diversi brevetti per le istruzioni militari sui
fazzoletti, tutte stampate su stoffe. Il primo fu il n. 10.774 dell'11
settembre 1885. Questo fu presentato da Thomas John Hayrnes di
Gibilterra al posto del Maggiore Fulton, impegnato nella campagna in
Sudan a quel tempo. Il brevetto è intitolato ``Un metodo migliore per
impartire le Istruzioni Militari alla Fanteria''. Questo un estratto
della spiegazione: `` Questa invenzione si riferisce a un metodo
migliore per impartire le istruzioni alla fanteria per quei rami
dell'arte militare di cui devono essere a conoscenza. A questo scopo io
stampo le istruzioni sopra il tessuto, in modo tale che possa essere
usato come fazzoletto tascabile. In questo modo l'attenzione
dell'utilizzatore sarà costantemente diretta all'argomento stampato sino
a quando l'oggetto in sé sia ancora utilizzabile e, in caso di perdita o
distruzione, è sicuro che venga sostituito non appena possibile,
rappresentando così un grande vantaggio rispetto al libro o l'opuscolo
fin qui adottato.'' Il fucile Martini Henry fu il principale soggetto di
questa prima edizione. Questo fucile a retro carica a colpo singolo fu
basato su di un meccanismo di otturatore disegnato dallo svizzero
Friedrich von Martini nel 1868. In aggiunta la sua canna fu rigata
seguendo le idee del fabbricante di pistole scozzese Alexander Henry.
L'arma da fuoco che ne risultò fu introdotta ufficialmente nell'esercito
Britannico nel 1874. La descrizione del brevetto di Fulton fu
accompagnata da bozzetti del fucile e alcuni altri esempi di
illustrazioni generali da poter stampare sul tessuto.

\begin{verbatim}
                                [pic]
    Frontespizio della Rivista che nel 2009 pubblicò una  ricerca  sulla
    nascita dei  fazzoletti  per  istruzione  militare.  Estensore  Dirk
    Ziesing, esperto certificato su antiche armi manuali.








                                [pic]

                 Fazzoletto militare britannico n°1









                [pic] Particolare del fazzoletto n°1

                                [pic]

                   Particolare del fazzoletto n°1
\end{verbatim}

{[}pic{]} Nella versione finale del fazzoletto troviamo il fucile
Martini-Henry al centro.Intorno ai relativi dettagli dell'arma e le
cartucce ci sono molte informazioni utili al soldato, come richiami e
segnali con la bandiera, l'alfabeto Morse e la perfetta sistemazione
dell'equipaggiamento della fanteria kit per l'ispezione della camerata.
Seguono alcuni motti per i soldati:

\begin{verbatim}
- L’obbedienza è il primo dovere per un soldato. La sua salvezza dipende
  dal  suo  sangue  freddo  durante  le  azioni.  Avvantaggiarsi   della
  copertura quando disponibile. Non fare mai fuoco senza  aver  ricevuto
  istruzioni, così come non prendere  iniziative  individuali.  Prendere
  bene la mira prima di far fuoco. Attaccare un  cavalleggero  dal  lato
  sinistro; un lanciere dal lato destro. Usare sempre grande giudizio in
  azione ed essere veloci con le munizioni
La  zona  esterna  della  stoffa  è  coperta  da  20  immagini  con  le
\end{verbatim}

corrispondenti spiegazioni. Un esempio che abbiamo è una scena di un
soldato mentre pone il suo elmetto sulla bocca del suo fucile mentre un
suo commilitone mira al nemico. Queste le frasi sottostanti: ``Il nemico
si può raggirare facilmente ponendo l'elmetto sulla cima di un bastone
oppure della baionetta. Questo può essere fatto senza difficoltà da un
soldato al riparo e spesso attirerà il fuoco del nemico costringendolo
ad esporsi. Così come ognuna delle 20 immagini hanno una spiegazione
sottostante, gli angoli forniscono spazio in più per informazioni
generali. Qui troviamo regole per la mira, istruzioni su come aver cura
del fucile e alla fine, la definizione delle ricompense per i tiri al
bersaglio. I riconoscimenti economici per i tiri migliori di soldati
semplici o caporalmaggiori sono descritti dettagliatamente.

IL SECONDO FAZZOLETTO BRITANNICO CON ISTRUZIONI

Il brevetto successivo è il n. 12801 e il richiedente scelse il titolo:
``Un nuovo o comunque migliorato metodo per impartire le istruzioni
militari ai soldati''. Nonostante ciò, la descrizione è praticamente
identica al precedente. La differenza più importante è il fatto che ora
il fucile che deve essere mostrato è menzionato in modo esplicito.Il
Times del 14 ottobre 1892 annunciò che il Tenente Colonnello Fulton
aveva pubblicato una versione migliorata del suo fazzoletto militare che
era stato approvato dalle autorità militari. Su questo è menzionato il
fucile Lee-Metford. Quest'arma fece parte dell'armamento della fanteria
Britannica dal 1890.

\begin{verbatim}
                                [pic]

                 Fazzoletto militare britannico n°2



                                [pic]

                   Particolare del fazzoletto n°2
\end{verbatim}

{[}pic{]}

Nel 1891 prese ufficialmente il nome dai nomi dei suoi disegnatori:
William Ellis Metford (1824-1899) che aveva messo a punto una
particolare scanalatura della canna, e James Paris Lee (1831-1904), nato
in Scozia ma residente in Nord America. Il suo lavoro per l'esercito
americano portò ai primi fucili a ripetizione. Insieme ad un sistema
unico di culatta fu utilizzato successivamente come nuovo fucile delle
forze armate Britanniche. Nel 1895 il nome fu cambiato in fucile
Lee-Enfield poiché il profilo della canna fu modificato seguendo le
proposte fatte dalla fabbrica di armi di Enfield. Questo accadde quando
la polvere da sparo (polvere nera propellente) fu sostituita da un
propellente senza fumo chiamato Cordite. Il Fucile Lee-Metford Mark I
aveva un tamburo da otto colpi di calibro 303. D'altro canto la
spiegazione del brevetto di Fulton fa riferimento anche ad una versione
successiva Mark II, dichiarando essere in uso alle Guardie a Cavallo. Il
nuovo fucile aveva una capacità di 10 colpi e fu introdotto nell'aprile
1893. L'adozione del brevetto di Fulton era datata 12 luglio 1892, ma la
descrizione finale fu fornita nella primavera del 1893. Questo significa
che fu un aggiornamento notevole. Così, sul fazzoletto, noi ne troviamo
due tipi, Mark I e Mark II, mostrati con i loro dati tecnici comuni. Le
immagini stampate erano già state rese disponibili nell'allegato del
brevetto. Il fazzoletto n.2 aveva versioni aggiornate delle 20 vignette,
ma il testo era rimasto invariato rispetto al n.1. Lungo il lato più
corto una riga riguardante le posizioni per l'esercizio con la baionetta
e la parata musicale.

\begin{verbatim}
                                [pic]

          Fazzoletto militare britannico n°3( della Marina)
\end{verbatim}

{[}pic{]}

Particolare del fazzoletto britannico della Marina

{[}pic{]}

Invece di una descrizione dettagliata dei premi per i bersagli di tiro
forniti sul n.1, c'è solo un semplice richiamo stampato sul n.2: ``Un
buon tiro può incrementare la tua paga da una sterlina a 20 all'anno.''
Come nuova caratteristica, gli angoli sono ora riempiti con le più
importanti decorazioni militari. Prima di tutto, abbiamo la croce
Victoria (Victoria Cross) nell'angolo più alto a sinistra. Fu emessa nel
1856 durante la guerra di Crimea. Fra i 111 premiati iniziali, il
giovane Charles Davis Lucas fu il primo a ricevere questo
riconoscimento. All'età di 20 anni nel 1854, egli fu un ufficiale
mediocre di seconda classe sulla bombarda HMS Hecla. Quando le flotte
alleate Britanniche e Francesi attaccarono le fortificazioni Russe a
Bomarsund (Baltico), Lucas raccolse una granata di mortaio russa con la
miccia accesa e la buttò fuoribordo. La seconda decorazione (in alto
nell'angolo a destra) è quello per meriti di servizio, assegnato ai
sottufficiali anziani dal 1845. Era abbinata ad una rendita vitalizia.
Il successivo (più in basso a destra) è la medaglia per anzianità e
buona condotta. Distribuito dal 1830 il disegno di questa medaglia vide
una lista di modifiche. Una di queste è la soppressione della giubba
dell'esercito di Hannover, quando divenne un regno indipendente nel
1837. All'inizio la medaglia LSGC veniva assegnata dopo 21 anni di
servizio effettuato per la fanteria e 24 per la cavalleria. Nel 1870 si
decise per la riduzione a 18 anni per tutti. L'ultima (in basso a
sinistra) è la medaglia per essersi distinti per la condotta sul campo.
Fu emessa all'inizio della guerra di Crimea nel 1854 e, successivamente,
assegnata a uomini arruolati, ma non divenuti ufficiali dell'Impero
Britannico. Si deve notare che le medaglie illustrate riportano il
ritratto della giovane Regina Vittoria. Più tardi il ritratto fu
aggiornato con i regnanti correnti, fino alla Regina Elisabetta II.

IL TERZO FAZZOLETTO BRITANNICO CON ISTRUZIONI

Il terzo brevetto fu presentato dal tenente colonnello Fulton in
pensione il 30 ottobre 1894. {[}pic{]} Il suo titolo era `'un metodo per
impartire istruzioni navali a marinai e uomini di mare'', e fu accettato
il 5 ottobre 1895, assegnandogli il n. 20.771. Il tondo al centro di
questo fazzoletto mostra un'ancora ed un salvagente. Sul salvagente sono
scritte le seguenti parole: `` L'Inghilterra si aspetta che ogni uomo
oggi faccia il suo dovere'', il messaggio inviato da Lord Nelson alla
flotta Britannica, prima della battaglia di Trafalgar. Analogo al
secondo fazzoletto della fanteria, abbiamo le due versioni del fucile
Lee-Metford presentato nel settore in alto a sinistra nel fazzoletto
della marina. Viene mostrata anche la pistola Werbley Mark I. 10.000
pezzi di questo revolver furono ordinati nel 1887 e messi in
circolazione nel 1890. La maggior parte di essi andò alla marina reale.
Il calibro fu inizialmente assegnato come 441 o 442, corrispondente al
diametro d'ingresso del mirino della canna. Più tardi il calibro 455 fu
usato al posto di quello disegnato. Le munizioni furono caricate con
polvere da sparo nera, sostituita nel 1894 dal propellente senza fumo
chiamato Cordite. Il revolver successivo fu introdotto nel giugno 1895
per esser utilizzato dall'esercito e dalla marina. Si può distinguere
dalla forma modificata dell'impugnatura e da un più grande percussore e
da un grilletto con la sicura. L'ultimo brevetto di Fulton non aveva
disegni allegati, ma l'informazione che doveva essere fornita fu
descritta nei minimi dettagli. Questo è valido anche per le decorazioni
marine visibili negli angoli, cioè la Croce Victoria (Victoria Cross) e
le medaglie per il notevole coraggio mostrato e per anzianità e buona
condotta. La medaglia Croce Victoria non fa distinzione tra l'esercito
ed il personale della marina. La medaglia per il coraggio fu istituita
nel 1885, assegnata sino al 1993 e, successivamente sostituita da una
croce (al valore). Dal 1831 la medaglia per l'anzianità e la buona
condotta (LSGC = long service good conduct) veniva assegnata dopo 21
anni di servizio. Nel 1848, il brevetto mostrato fu introdotto
caratterizzato da un vascello a 3 alberi. Nel 1874 il periodo di
servizio fu accorciato a 10 anni, ma più tardi fu aumentato ancora a 15
anni. L'oggetto che doveva esser mostrato nel quarto angolo non era
ancora stato definito quando il brevetto fu specificato. Nelle
descrizioni era semplicemente chiamato ``medaglia''. Sul fazzoletto
stampato per la marina, divenne quello della Royal {[}pic{]}

Human Society. Il loro motto latino ``Lateat Scintilulla Forsan'' può
essere letto sulla stoffa. Questo significa che una piccola scintilla
potrebbe essere nascosta, riferendosi all'ultima scintilla di vita che
può essere recuperata nelle persone annegate, con una respirazione bocca
a bocca adeguata. L'insegnamento di ciò fu il compito della RHS (Royal
Humane Society) fondata nel 1774 dai medici Dr.~Hawes e Dr.Cogan. Un
anno più tardi, la medaglia in argento mostrata, fu emessa come il più
alto riconoscimento per le azioni di salvataggio in acqua. Dal 1837, fu
disponibile anche una versione in bronzo, e nel 1873, fu conferita la
prima medaglia d'oro chiamata'' Medaglia Stanhope'', in onore dell'uomo
che salvò vite in parecchie situazioni d'emergenza. E' ancora conferita
una volta all'anno per le azioni più ragguardevoli. Sul fazzoletto della
marina, oltre a queste decorazioni, vengono date informazioni utili.
Sulla destra, possiamo trovare segnali a bandiera così come l'alfabeto
Morse. Venti squilli di tromba sono posizionati intorno ai bordi. Stemmi
caratteristici vengono mostrati insieme alle istruzioni su come fare il
saluto militare correttamente. Nella zona in basso, possiamo vedere
illustrazioni su come effettuare gli esercizi con la sciabola. Come il
fazzoletto per la fanteria, anche quello della marina fornisce le
indicazioni per disporre l'abbigliamento militare della marina per
l'ispezione. Le linee guida stampate riferite al soccorso delle persone
che stanno annegando possono essere interpretate come un tributo alla
Royal Humane Society. Queste erano accompagnate da ulteriori istruzioni
su come fare un laccio emostatico per un braccio in caso di emorragia.
Alcuni proverbi di uomini di mare completano l'informazione. Qui di
seguito un esempio: ``Se le nuvole si addensano fitte e veloci, stare
attenti e guardare vele e alberi, se proseguono per la loro strada, alza
le rande di nuovo''. Per alcuni aspetti, il fazzoletto per la marina
differisce dai precedenti predisposti per l'esercito. Prima di tutto ha
una cornice blu scuro invece di quelle rosse ed il nome di due autori
viene fornito qui. {[}pic{]}

La scritta ``Fulton-Airey'' significa che l'ufficiale di fanteria Fulton
aveva trovato un compagno per poter avere accesso alla conoscenza di
questioni della marina. Sebbene finora non sia ancora stata trovata una
prova, solo un ufficiale della marina reale fu preso in considerazione
in quel particolare periodo, per esempio alla fine del 19° secolo. Il
suo nome era Frederick Wilkin Iago Airey (1861-1922). Nel 1877 iniziò la
sua carriera navale come semplice assistente. Dal 1882, egli fu
assistente tesoriere su diversi navi della Marina Reale. Nel 1893, fu
promosso tesoriere sul vascello Magpie, nel 1899 fu Tesoriere del
personale, nel 1901 fu Tesoriere della flotta. Durante la Grande Guerra,
Airey fu nella Guardia Costiera dal 1914 al 1916. Nel 1917, la sua
promozione finale fu Tesoriere Capo, pari al capitano della nave. Si
ritirò durante l'ultimo comando del vascello Kingfisher nel 1918.
Frederick Airey potrebbe benissimo aver incontrato Carrè Fulton mentre
entrambi prestavano servizio in Egitto ed in Sudan. Egli ricevette anche
la medaglia per questa campagna, così come la Khedive Star nel 1882. Dal
luglio 1884 al febbraio 1885, durante la campagna di Suakin,
l'assistente tesoriere Airey ebbe il compito dei rifornimenti fatti con
trasporti navali. In questo contesto, ci fu un fatto di notevole
importanza. Con l'autorizzazione dell'Ammiragliato, Airey fece bozzetti
delle difese di Suakin, che furono successivamente pubblicati su alcuni
giornali inglesi. Ciò significa che un ufficiale della marina dette
prova di avere doti nella grafica, che potevano essere molto utili per
creare l'impaginazione per il fazzoletto con le istruzioni di Fulton.
Ultima cosa, ma non meno importante, Airey pubblicò un libro illustrato
di fiabe intitolato ``Pidgin code Inglis e altri''. Questo conteneva
storie dalle colonie Britanniche scritte in inglese Pidgin 2 (vedi a
pag. 36). La collaborazione fra Fulton e Airey fornì condizioni perfette
per disegnare un istruzione su stoffa per marinai e uomini di mare, così
come descritto nel brevetto. Dalle informazioni del Museo Nazionale
Marittimo di Greenwich, questi fazzoletti furono distribuiti ai giovani
marinai sino al 1899. I fazzoletti illustrati sparirono quando le
istruzioni divennero obsolete, come gli esercizi con la sciabola nel
1901. Essi furono sostituiti da tessuti di solo colore blu, che rimasero
in uso sino al 1938. Solo una annotazione in più in riferimento alla
misura dei fazzoletti Britannici con le istruzioni che sono leggermente
diverse. Il primo misura 30 x 26 pollici, mentre il secondo è più
piccolo e misura 24 x 20 pollici. Il fazzoletto della marina più largo
misura 33 x 27 pollici.

L'ORIGINE DEL FAZZOLETTO CON ISTRUZIONI MILITARI

In realtà le istruzioni militari stampate su stoffa non furono inventate
da Carrè Fulton, ma da un omologo francese di nome Pierre Perrinon.

\begin{verbatim}
                                [pic]
\end{verbatim}

Egli fu un ufficiale della fanteria francese, e fu il primo che si
attivò, nel 1874, per un brevetto che descriveva un fazzoletto che
potesse dare istruzioni sia militari che civili. Questo fu la base per
una serie di dieci differenti fazzoletti con le istruzioni fatti per i
soldati francesi, che contenevano soggetti riguardanti la fanteria, la
cavalleria, l'artiglieria e del genio (soldati genieri). Il brevetto
francese n. 105.411 ebbe una validità di 15 anni. Così, parlando in modo
schietto, Fulton non avrebbe dovuto ottenere un suo proprio brevetto nel
1885, ma a quei tempi le regole riguardanti i brevetti non erano sempre
applicate come oggi. Secondo alcune versioni, Fulton copiò direttamente
il lavoro originale di Perrinon. Egli prese le illustrazioni d'insieme
delle scene di battaglia e di servizi su campo dal fazzoletto francese
n° 1. Il seguente esempio mostra che la descrizione inglese è pressoché
una traduzione letterale del testo originale francese: ''Un soldato che
non ha più cartucce ed è costretto a combattere contro un cavaliere si
deve mantenere alla sua sinistra, lato su cui un cavaliere difficilmente
è armato contrariamente alla destra, e non è più in grado di guidare la
sua cavalcatura; se il soldato è costretto a combattere di fronte, non
dovrà mai colpire il cavallo al torace ma al collo o alla testa, al che
il cavallo si impenna. I campi nei quali le coltivazioni sono sostenute
da fili di ferro sono dei veri ostacoli per i cavalieri.'' Sul primo
fazzoletto di Fulton leggiamo lo stesso in inglese: ``Un soldato che non
ha munizioni ed è obbligato a combattere contro un uomo a cavallo, deve
sempre tenersi alla sua sinistra essendo questo è il lato su cui è meno
efficiente. Se viene ferito sulla sua sinistra, non riesce più a
controllare la sua cavalcatura. Se il soldato è obbligato a combattere
frontalmente, mai ferire il cavallo al torace bensì alla testa o al
collo. I campi con siepi avvolte sono eccellenti ostacoli per la
cavalleria''. La corrispondente immagine francese mostra un fante
all'attacco di un militare a cavallo, che indossa un elmetto tedesco a
punta. Da parte sua, Fulton fece combattere un soldato inglese a piedi
contro un ussaro, dall'aspetto identico al francese! La frase del testo
di Perrinon finisce con il richiamo: ``È nostro dovere per tutti
difendere contro lo straniero il suolo dove noi siamo nati. Se tutti i
francesi si sentono coinvolti quando la patria è in pericolo, facciamo
in modo che tutti questi volontari diventino dei veri soldati. Lavoriamo
dunque senza sosta, istruiamoci. Non dimentichiamo mai che gli occhi di
nostra madre, gli occhi di DIO, il Diritto, la Giustizia prevalgono
sulla Forza. {[}pic{]} Dobbiamo essere pazienti, laboriosi, uniti
nell'amore sacro per la patria e ben presto nostri cari, la ben amata
Francia si risolleverà più bella, più gloriosa che mai. VIVA LA
FRANCIA!!!'' Il messaggio patriottico di Fulton risuona: ``È dovere di
ogni uomo difendere il paese dove è nato contro ogni nemico. Quindi
lasciate che ogni inglese sia un vero soldato. Lasciateci lavorare senza
sosta. Lasciateci istruire. Non dobbiamo mai dimenticare che gli occhi
di DIO, il Diritto, la Giustizia sono dei valori più che consolidati.
Lasciateci essere pazienti e industriosi, lasciateci essere uniti dal
segreto amore per il nostro paese e, da ultimo, lasciateci ricordare
sempre (parole di Nelson) che l'Inghilterra si aspetta che ogni uomo
faccia il suo dovere. LUNGA VITA ALLA REGINA.'' Ad eccezione dell'ultima
esclamazione, questo testo è di nuovo una semplice traduzione dal
francese all'inglese ma, laddove Fulton cita Nelson e osanna la regina,
Perrinon risolleva l'umore dei suoi compatrioti negli anni successivi al
disastroso esito della guerra franco-prussiana ``\ldots{}ben presto la
nostra cara e benamata Francia si risolleverà ancora più bella, più
gloriosa che mai.''

CONCLUSIONE

Ultimamente, i fazzoletti militari d'istruzione dell'Impero Britannico
non hanno trovato vasta affermazione come i loro equivalenti
continentali. Le autorità militari francesi ne sponsorizzavano persino
l'acquisto, mentre le reclute inglesi se li dovevano comperare a proprie
stesse spese. Da ciò, esemplari conservati in buone condizioni sono di
alquanto difficile reperimento e anche molto costosi, in particolare
quelli della marina. Una collezione di tutti e tre i fazzoletti di
istruzione britannici (esercito, marina, aviazione) sarebbe una grande
scoperta persino per un appassionato di oggetti militari.

\begin{verbatim}
                            Bibliografia
\end{verbatim}

Hayden, J.L.: Fulton's Military Handkerchief, Antique Arms \& Militaria
Bogle, M.: Mouchotr d'instruction militaire, Journal of the Australian
War Memorial, 1983, n° 2 Chamberlain, W.H.J. and Taylerson, A.W.F.:
Revolvers of British Services, 1854-1954, Museum Restoration Service,
1989 Informazioni fornite dal museo reggimentale della Durham Light
Industry, Durham. Sull'autore

Dirk Ziesing è un esperto certificato su antiche armi manuali e autore
di numerosi articoli. Attualmente sta lavorando a un libro dedicato ai
fazzoletti d'istruzione militare.

\begin{verbatim}
                       Note della traduttrice
\end{verbatim}

Mohammed Ahmed Mahdi (ca. 1844 - Omdurman 1885) è stato protagonista
della resistenza anticoloniale africana. Fra i molti leader
rivoluzionari, che all'interno della tradizione musulmana si
proclamarono e furono venerati,il sudanese Mahdi (Guidato da Dio) occupa
un posto particolare. Eccezionale fu il riscontro che seppe creare fra i
capi e il popolo. Già nel 1883 la sua Mahdijja controllava gran parte
del Sudan centrale, nel 1884 sconfiggeva un consistente corpo di truppe
inglesi e nel 1885 prendeva Khartum. Mori quando il Sudan era ormai in
sua mano. Il ``Pidgin'' è un idioma derivante dalla mescolanza di lingue
di popolazioni differenti, venute a contatto a seguito di migrazioni,
colonizzazioni, relazioni commerciali.

Capitolo II

\begin{verbatim}
                            I fazzoletti
                       per istruzione militare
                              in Italia
                     Produzioni della stamperia
                           De Angeli Frua
                              1884-1918
                      Prima parte - Generalità



                           di Dirk Ziesing
\end{verbatim}

NOTE SULLA TRADUZIONE DALL' ARTICOLO ORIGINALE IN LINGUA TEDESCA

I dati demografici del presente contributo riflettono anche quanto
compare direttamente sui fazzoletti analizzati; essi possono quindi
discordare da quelli ufficiali. Il testo è stato redatto prima degli
avvenimenti che hanno modificato l'assetto istituzionale di molti stati
della sponda meridionale del Mare Mediterraneo. {[}pic{]}

De Angeli-Frua

La stamperia italiana per fazzoletti d'istruzione era a Saronno, una
cittadina a nord di Milano. Era così chiamata dal nome di Ernesto De
Angeli (1849-1907). Egli aveva ricevuto la sua formazione dal barone
Costanzo Cantoni, il fondatore dell'industria tessile in Italia. La sua
filanda e cotonificio era a Castellanza dal 1845. Nel 1872 fu fondata la
``Società Ernesto De Angeli \& C.'' e nel 1878 si aprì la stamperia
milanese di Cantoni. De Angeli raggiunse una grande reputazione e nel
1895 fu nominato Senatore. In questo contesto si deve citare anche
Giuseppe Frua (1855-1937). Iniziò diciassettenne ad imparare la
tessitura in un'impresa tedesca. Tornato in Italia trovò un primo
impiego nel Cotonificio Caprotti. Nel 1875 entrò nella ditta di Eugenio
Cantoni, figlio e successore del già menzionato barone. Nel 1879 Frua
assunse la direzione commerciale della sede di Castellanza. Nel 1883
sposò Anna De Angeli e divenne Procuratore della ditta del cognato. Poi
nel 1890 divenne socio della fabbrica tessile dei fratelli Banfi a
Legnano (Anonima Frua \& Banfi), nel 1896 Ernesto De Angeli e Giuseppe
Frua unirono le loro filature, cotonifici e stamperie in una società
denominata De Angeli-Frua. All'inizio del XX secolo sperimentò anche la
fabbricazione della seta. A questo fine fu acquistato un terreno sul
lago d'Orta e furono sostituiti i vigneti ed i frutteti con piante di
gelso, delle cui foglie vi era necessità per l'allevamento dei bachi da
seta. Inoltre, in quel tempo fu acquistato sul Lago Maggiore un
palazzetto del 18° secolo a scopo di rappresentanza. Questa villa De
Angeli Frua esistette, come parte della ditta a Milano, fino ad oggi.
Nel 1937 si contavano 5 fabbriche con 11000 addetti. Lo stabilimento di
Gerenzano, vicino a Milano, fu un fiore all'occhiello, nel quale intorno
all'abitazione del Direttore stavano edifici per uffici, mensa,
scolastici ed un asilo aziendale. In seguito ai problemi economici
dell'industria tessile europea nei confronti della concorrenza asiatica,
chiuse nel 1965.

I fazzoletti militari Itaiani

Sguardo d'insieme

Lo stampatore di fazzoletti milanese si dedicò anche ai fazzoletti
d'istruzione militare. Il materiale cartografico è particolarmente
importante, perché fu inserito in Italia in uno schema numerico. Così
nel 1884 nacque in Italia il primo fazzoletto ad impronta militare.
Seguirono fazzoletti indirizzati alla tecnica militare ed infine con
temi geografici.

Il fazzoletto italiano n°1 (vedi pag. 40)

La mappa riprodotta sul fazzoletto nel formato 64 x 58 cm mostra lo
stivale italiano, le grandi isole di Sardegna e Sicilia e il resto
dell'arcipelago.

{[}pic{]}

\begin{verbatim}
                         Carta d’Italia 1884
                  Fazzoletto militare italiano n°1
                    Stamperia De Angeli – Milano
\end{verbatim}

{[}pic{]}

\begin{verbatim}
                  Fazzoletto militare italiano n°2
                    Stamperia De Angeli – Milano
\end{verbatim}

{[}pic{]}

Oltre alle strade di 1a, 2a e 3a classe e le linee ferroviarie,
compaiono i numeri dei dodici corpi d'armata dell'esercito e la
posizione dei posti di comando e delle fortificazioni. II titolo
stampato recita ``CARTA DIMOSTRATIVA DELLE CIRCOSCRIZIONI MILITARI DEL
REGNO D'ITALIA 1884'' (mappa panoramica delle circoscrizioni militari
del Regno d'Italia nel 1884). Sulla circonferenza esterna sono
rappresentati gli stemmi dei distretti italiani, con i dati della
popolazione. Negli angoli e al centro dei lati sono rappresentate le
maggiori città: Roma, Firenze, Napoli, Genova, Torino, Palermo, Milano e
Venezia. Esse sono allineate in ordine alfabetico e in senso antiorario
da Alessandria a Vicenza.

Il fazzoletto italiano n°2 (vedi pag. 41)

II secondo fazzoletto della serie italiana descrive il modello del
fucile 1870/87 Nel centro figura il testo ``Fucile Modello 1870/87'' e
sul bordo ``FAZZOLETTO MILITARE N° 2 - (PRIVATIVA INDUSTRIALE)'' e
``STAMPERIA E. DE ANGELI \& C - MILANO''. L'occasione fu quella, nel
1887, dell'ammodernamento del modello Vetterli del 1870. Un fazzoletto
relativo a questo modello non è disponibile. Esso era stato prodotto
dalla ditta Rolffs,(vedi pag. 44) come modello dei fazzoletti di
istruzioni per l'Austria-Ungheria. All'atto del miglioramento, le armi
esistenti sono state dotate di un caricatore multiplo. Il caricatore
utilizzato per quattro cartucce era stato sviluppato dal capitano
d'artiglieria Giuseppe Vitali. Seguendo lo stesso principio un anno
dopo, furono aggiornate le armi olandesi Beaumont. Dal momento che le
armi italiane sono state adattate in larga misura, fucili Vetterli in
condizioni originali sono abbastanza rari da trovare. II caricatore per
l'alimentazione è dotato di un'apertura sul fondo. Per compensare
l'indebolimento del fusto, una lastra di metallo è stata aggiunta al
legno. Inoltre, si aggiunse una guida per il percussore. Il coperchio di
protezione del carico superiore è stato rimosso e sostituito da un
anello metallico. Infine, c'è stata una variazione della sicura e della
baionetta. Il mirino a quadrante elaborato da Vecci rappresenta una
modifica che è stata attuata nel 1881 sul modello del 1870.

Il fazzoletto italiano N°3 (vedi pag. 45)

I fucili 1887 sono stati sostituiti pochi anni dopo con un nuovo fucile
a ripetizione. Inoltre, ci fu ancora una volta un fazzoletto di
istruzioni. E' intitolato ``Fucile Modello 1891'' e porta sul bordo
orizzontale il testo ``FAZZOLETTO MILITARE n ° 3 - (PRIVATIVA
INDUSTRIALE)'' e ``STAMPERIA E. DE ANGELI \& C -- MILANO''. Il nuovo
fucile italiano è stato introdotto in data 29 Marzo 1892. Si tratta di
un caricatore basato sul principio Mannlicher. Ferdinand von Mannlicher
si era aggiudicata una fornitura di 300.000 lire. L'otturatore era stato
progettato nell'arsenale di Torino da Salvatore Carcano (1827- 1903),
tenente colonnello e costruttore di armi. A volte viene richiamato in
questo contesto, anche il nome di Gustavo Parravicino, Presidente della
Commissione per la sperimentazione della nuova arma. La robusta arma
aveva una capacità di sei pallottole. Il calibro relativamente piccolo
di 6,5 x 52 mm, da un lato, ha offerto il vantaggio a ciascun soldato di
poter portare con sé una maggiore quantità di munizioni.

{[}pic{]}

\begin{verbatim}
                      Fazzoletto militare n° 2A
                            Rolffs & Co.
\end{verbatim}

{[}pic{]}

\begin{verbatim}
                  Fazzoletto militare italiano n°3.
                    Stamperia De Angeli – Milano
\end{verbatim}

{[}pic{]}

\begin{verbatim}
D'altra  parte,  sono  state  generalmente   criticate   le   inadeguate
\end{verbatim}

prestazioni balistiche. Inoltre, la cartuccia Carcano che era ancora
mantenuta, era un proiettile con punta arrotondata, mentre in altre
nazioni da tempo si erano convertiti i proiettili. Un altro svantaggio è
la lunghezza del primo modello introdotto per la fanteria. Ha un mirino
con una scala mobile 450-2000 metri. La lunghezza di 1,60 m con
baionetta innestata, riflette ancora il concetto che un soldato dovrebbe
essere in grado di attaccare anche un cavalliere per avere successo. Per
la guerra di trincea, che si è sviluppata durante la Prima Guerra
Mondiale, ciò è stato di molto ostacolo. Così si aggiunsero alla
famiglia Carcano versioni ridotte. Il ``Moschetto Cavalleria'' è stato
introdotto nel 1893 per la cavalleria e poi utilizzato dai paracadutisti
italiani. Questo modello è stato anche dotato di una baionetta
pieghevole, che si trova sotto la canna quando non in uso e con la punta
in una scanalatura. Alla fine del 1897 apparve anche il ``Moschetto
Truppe Speciali''. Come dice il suo nome è stato prodotto per unità
speciali, come le truppe di telecomunicazioni, cannonieri e piloti, ma
anche le truppe d'assalto ed i carabinieri. Le versioni accorciate
sparavano le munizioni stesse di quelle lunghe, tuttavia, la portata era
ridotta fino a un massimo di 1500 metri. Tra il 1924 e il 1928 i fucili
di fanteria esistenti sono stati accorciati alla lunghezza di
quest'arma. Di questi, alcuni finirono dopo la Seconda Guerra Mondiale
usati come arma di servizio dalla polizia bavarese. Nel complesso, in un
periodo superiore ai 50 anni furono usate più di quindici varietà di
armi con sistema Carcano. Queste includono calibri allargati a 7,35 mm x
51, nonché cartucce tedesche calibro 7,92 x 57 mm e infine la riduzione
dal 5,5-6,8 mm per l'organizzazione giovanile fascista al tempo di
Mussolini. Una notorietà fu acquisita dal fucile Carcano il 22 Novembre
1963, quando Lee Harvey Oswald, a Dallas, uccise il presidente americano
John F. Kennedy. Usò un modello 1891/38 ``Fucile Corto'' con il numero
di serie C 2766. Interessante è la diversità delle cartucce raffigurate
sul fazzoletto- Carcano. Ci sono, oltre alla normale cartuccia con
copertura di nickel (``a pallottola'') quelle per esercitazione (``da
esercitazioni'') ed una cartuccia a salve (``a salve''). Una
caratteristica particolare è il massimo di undici proiettili singoli in
un sottile involucro di ottone (``a mitraglia''), che è stato utilizzato
con una carica propellente ridotta nelle mischie. Infine c'erano le
munizioni per il tiro al bersaglio (``per Tiro a Segno''), con carica
propellente ridotta e un nucleo in piombo più piccolo. Per mantenere la
lunghezza completa il volume rimanente era riempito con sabbia. La
conclusione dei disegni in sezione mostra una cartuccia con punti di
rottura (``Frangibile''). Fra il proiettile e un piccolo cilindro di
alluminio il terminale è pieno di sabbia. La necessità di tale una
speciale cartuccia si era evidenziata, dal momento che nel tiro al
bersaglio con normali proiettili essi rimbalzavano pericolosamente
colpendo le pareti. Queste cartucce sono mostrate nell'area inferiore
del fazzoletto. Sui fazzoletti italiani si possono anche trovare le
immagini e le spiegazioni delle varie formazioni di fanteria. {[}pic{]}

Negli angoli stanno quattro rappresentanti di diverse unità che mostrano
i passaggi per il caricamento ed il fuoco. La scritta ``caricat'' è la
forma abbreviata di ``caricate'' e significa ``Carica!'' La sequenza
inizia nell'angolo in basso a sinistra con l'apertura della caricatore.
Ne consegue l'introduzione dei proiettili nel caricatore. Il terzo passo
descrive la chiusura e la quarta immagine con la scritta``punt'' nella
parte superiore il puntamento dell'arma. Inoltre, sono illustrate le
sedici posizioni generali uso dell'arma in un cerchio intorno al centro.
Nel fazzoletto per fucile 1870/87 le figure sono tutte in uniforme della
fanteria di linea, mentre nei fazzoletti successivi i gruppi di divise
sono diversi. All'interno dei gruppi le uniformi sono da campo, da
deposito, da parata e coloniali. Le figure da 1 a 4 rappresentano la
``Brigata Granatieri di Sardegna'' A seguire da 5 a 9 la ``Fanteria di
Linea'', da 10 a 12 gli ``Alpini'' e infine da 13 a 16 i
``Bersaglieri''. II primo reggimento dei Granatieri di Sardegna fu
istituito nel 1659 a Torino come guardia dei duchi di Savoia. II secondo
seguì in Sardegna. Ci si era ritirata la casa regnante dei Savoia,
mentre l'Italia era sotto l'influenza di Napoleone Bonaparte. I
Granatieri della Guardia e i Cacciatori sono stati successivamente
raccolti in una brigata. Dopo il 1861 ci furono altre tre brigate, ma
poi confluirono nella fanteria di linea. La brigata è stata abolita nel
1919 in Italia e i due reggimenti Guardie della 21 Divisione di
fanteria, furono assegnati a Roma, che poi hanno portato il nome di
``Granatieri di Sardegna''. Fin dalla sua costituzione, la Guardia ha
partecipato a molte battaglie importanti. Nel 1848 i Savoia inizialmente
hanno combattuto con successo contro gli austriaci comandati dal
maresciallo Radetzky. Il 30 Maggio re Carlo Albèrto prese il comando
personale durante la battaglia di Goito, con l'esclamazione ``A me le
Guardie!''. Queste parole sono state incorporate nel motto del
reggimento. Durante l'offensiva di primavera del 1916 i granatieri hanno
resistito contro le truppe austro-ungariche, e dopo aver esaurito le
munizioni a Monte Cengio, si buttavano giù con i loro nemici in
combattimento ravvicinato, dalle pareti di roccia. Le immagini
successive mostrano di squadre di fanteria di linea (``Fanteria di
Linea''). Complessivamente ci sono stati 96 reggimenti. Fra i più
antichi sono l'11 e il 12, che risalgono al 1619, in Piemonte. Le
ordinanze del 1903 per ciascuno delle dodici corpi di armata
dell'esercito, prevedevano 27 battaglioni di fanteria con una forza
totale di 31.000 uomini. Gli ``Alpini'' formano una speciale forza di
fanteria. Istituiti nel 1872 per assicurare la frontiera alpina, sono
considerati i primi cacciatori di montagna del mondo. Inizialmente
c'erano solo quindici compagnie, che crebbero rapidamente alla forza di
un reggimento. Allo scoppio della seconda guerra mondiale c'erano 50
battaglioni, che sono stati aumentati a 88. In contrasto con la fanteria
di linea, le truppe di fanteria di montagna sono reclutate dalla loro
zona di provenienza in modo a essere al corrente delle condizioni
alpine. Tra il 1915 e il 1918 furono impiegati in sanguinose battaglie
sul fiume Isonzo (oggi in Slovenia) e sul Piave contro gli austriaci.
Ebbero fama ulteriormente i battaglioni da sci dell'Adamello, una cima
delle Alpi Meridionali. {[}pic{]}

I ``Bersaglieri'' sono le unità d'elite della fanteria italiana. Il
termine deriva dalla parola ``Bersaglio''. Erwin Rommel di loro ha detto
dopo l'esperienza nell'Afrika korps: ``II soldato tedesco è ammirato nel
mondo, ma il soldato tedesco ammirò il Bersagliere italiano''. Questo
corpo si era formato nei 1836 in Piemonte, dopo la riforma dell'esercito
del 1831 come di fanteria leggera e rapida. Il modello era il cacciatore
francese a piedi, che operava in piccole unità indipendenti. Il loro
compito era la perlustrazione, la sorpresa e l'assalto. Operando i
bersaglieri come tiratori, che erano particolarmente considerati dagli
ufficiali nemici. Prima della fondazione degli Alpini, i Bersaglieri
sono stati utilizzati anche in montagna. Fino al 1843, furono schierate
quattro compagnie del Primo Battaglione. Dal 1871, ci sono state 36
compagnie raggruppate in sei reggimenti. Nel 1910 è stato allestito un
battaglione ciclista. Nella prima guerra mondiale, i bersaglieri
operavano in due divisioni speciali. I severi requisiti si riflettevano
nel reclutamento, e i bersaglieri hanno sempre goduto della massima
reputazione da parte della popolazione. Con i loro cappelli, differivano
significativamente dal resto della fanteria. Il cappello indossato dal
1871, è decorato sul lato destro con piume di gallo cedrone. La coccarda
verde e bianco reca come un distintivo di ottone sbalzato fucili e
granate con il numero del reggimento. Anche sull'elmo d'acciaio 1916 i
bersaglieri misero le loro piume. A proposito del copricapo dei
Bersaglieri vi è ancora da fare un'altra menzione speciale. E 'il fez
rosso, che è stato portato nell'uniforme al campo, o quando fa freddo
anche sotto il cappello. Questo copricapo orientale fu donato agli
italiani nella guerra di Crimea nel 1855, dai turchi alleati, in
riconoscimento del loro coraggio nella battaglia di Chernaya. Sul
fazzoletto di istruzioni n. 3 il Fez figura al punto 14 (``in spalla il
suo fucile!''). Anche le decorazioni dei militari italiani si trovano
sui fazzoletti. La Medaglia (``al valor militare'') che porta sulla
parte anteriore la croce di Savoia, esisteva sin dal 1793, ma è stata
rinnovata nel 1833. Questa medaglia è stata assegnata in bronzo, argento
e oro e indossata con un nastro azzurro. Una prima medaglia
commemorativa mostra l'immagine di Re Vittorio Emanuele II (1820 -
1878). E 'stata creata per commemorare le guerre di indipendenza.
Vittorio Emanuele veniva dalla Casa di Savoia ed era già re di
Piemonte-Sardegna dal 1849 al 1861, prima di diventare il Re del nuovo
stato nazionale italiano. Come guida del movimento di unificazione
italiana, è stato alleato con la Francia sotto l'imperatore Napoleone
III. insieme hanno sconfitto l'Austria nel 1859 nel nord Italia, e la
Lombardia è stata conquistata. Nella sanguinosa battaglia di Solferino,
il 24 Giugno 1859 l'imprenditore svizzero Henri Dunant prese la
decisione di fondare la Croce Rossa. Il numero di circa 40.000 feriti in
battaglia raddoppiò nei giorni seguenti per malattie, e quasi nessuna
cura medica a disposizione. Per iniziativa di Dunant a Ginevra nel 1863
fu istituito il ``Comitato internazionale di soccorso ai feriti in
battaglia''.

{[}pic{]}

\begin{verbatim}
          Fazzoletto militare italiano n°4 per Cavalleria.
                Stamperia E. De Angeli &Co. – Milano
\end{verbatim}

{[}pic{]}

\begin{verbatim}
                     Fazzoletto per armi pesanti



                                [pic]

Nel  sud  Italia,  nei  frattempo,  l'italiano  Giuseppe  Garibaldi,  ha
\end{verbatim}

scacciato con i suoi guerriglieri nel 1860/61 i Borboni francesi di
nascita dal trono del Regno delle Due Sicilie. in un plebiscito, il
popolo di questa vasta area della Sicilia e di Napoli si espresse per
l'unione al Regno d'Italia. Da allora in poi solo il Veneto rimase sotto
il dominio degli Asburgo, e Io Stato della Chiesa di Papa Pio IX era
sotto la protezione francese. Vittorio Emanuele, però, non si fermò
nella lotta per l'indipendenza d'Italia. Così lasciò la città di Nizza e
la Savoia alla Francia in cambio del sostegno avuto contro l'Austria.
Nella guerra austro- prussiana nel 1866 l'Italia si schierò con la
Prussia e, quindi, ottenne la zona intorno a Venezia. Infine, nel 1870
anche lo Stato Pontificio è stato occupato dalle truppe italiane, di
stanza là dopo che le truppe francesi erano state ritirate per
proteggere il fronte della guerra franco-tedesca. Roma divenne capitale
dell'Italia unita. II fazzoletto riporta anche una medaglia con la
effigie del re Umberto I (1844 - 1900) per l'unità del paese (``Unità
d'Italia''). Il figlio e successore di Vittorio Emanuele II, re d'Italia
nel 1878 e duca di Savoia. Morì a causa di un attentato che fu
perpetrato per colpire il re che aveva decorato un generale che aveva
aperto il fuoco contro manifestanti disarmati. La conclusione delle
medaglie mostra una effigie di Vittorio Emanuele III. (1869 - 1947). Non
fu solo dal 1900 al 1946 Re d'Italia, ma anche Imperatore d'Etiopia
(1936 - 1941) e Re d'Albania (1939 - 1943). La medaglia commemorativa è
stato coniata nel settembre 1913 e ricorda la guerra di Libia (retro:
``Libia''). E dotata di un nastro blu e rosso a strisce. In seguito alla
campagna italiana in Libia in seguito ha avuto luogo la guerra
italo-turca nel 1912/13, che portò alla annessione della colonia
italiana in Africa del Nord. A tal fine, fu presentato anche uno
speciale fazzoletto geografico. Per completare i riconoscimenti ai
militari italiani fù coniata la croce per lunghi periodi di servizio
(``Croce d'anzianità''). Erano premiati in argento per 16 anni, in oro
per 25 anni e in oro con una corona per 40 anni. L'illustrazione della
medaglia commemorativa del 1913, permette di datare il fazzoletto
italiano n° 3 al periodo di poco precedente lo scoppio della prima
guerra mondiale. E' il più giovane membro della famiglia di fazzoletti
di istruzione europea, perché tutte le altre nazioni avevano completato
la loro edizione di fazzoletti molto prima.

{[}pic{]}

\begin{verbatim}
                         Carta d’Italia 1912
                  Fazzoletto militare italiano n°5
                Stamperia E. De Angeli Frua – Milano
\end{verbatim}

{[}pic{]}

\begin{verbatim}
                         Carta d’Italia 1918
                  Fazzoletto militare italiano n°6
                Stamperia E. De Angeli Frua – Milano


                                [pic]
\end{verbatim}

Il fazzoletto italiano con carta geografica del 1912 (vedi pag. 56)

La versione del fazzoletto del 1912 reca lo stesso titolo della prima
esecuzione del 1884. Con 64 x 58 cm è anche Io stesso formato. Gli
stemmi sono riportati con le seguenti eccezioni: invece di Porto
Maurizio c'è Imperia, poiché la città della Riviera era stata riunita
con la vicina Oneglia. Rovigo, era ormai in ordine alfabetico
correttamente ordinata dietro Reggio Calabria e Reggio Emilia. inoltre,
sulle mappe dell'area interna, sopra il titolo, erano aggiunti cinque
stemmi. Sembra che nel 1884 i distretti di Campobasso, Grosseto, Lecce,
Potenza e Siracusa fossero semplicemente stati dimenticati. In tutti i
distretti, i dati demografici sono stati rettificati. Per esempio, è
aumentata la cifra per Roma da 864 851 a 1.343.392. E' anche riportata
la popolazione totale dell'Italia del 1912, con un valore di 34.978.634,
con una superficie di 268.682 km2. Un complemento essenziale mostra le
rappresentazioni geografiche delle colonie italiane della fine del 19 °
Secolo. Due mappe in dettaglio mostrano la scritta ``POSSESSIONI
D'OLTREMARE COLONIA ERITREA - SOMALIA -- TRIPOLITANIA''. Sul Mar Rosso,
la colonia italiana d'Eritrea era stata creata nel 1890. Dopo la seconda
guerra mondiale questa zona era una federazione con l'Etiopia. Solo dopo
un decennio di guerra, l'Eritrea ha ottenuto l'indipendenza nel 1993.
Nel Corno d'Africa, sotto il Golfo di Aden vi era la colonia della
``Somalia Italiana''. Gli italiani avevano preso nel 1880 la proprietà
di questa zona. Nel 1908 Mogadiscio era diventata la capitale della
colonia italiana di nuova costituzione. Nel 1960, la Somalia italiana e
la Somalia britannica furono unite come unico stato indipendente con
nome Somalia. {[}pic{]} Sul Mediterraneo la Tripolitania deriva dal nome
della capitale Tripoli. Questa zona fu annessa nel 1911-1913 sulla scia
della guerra italo-turca e della guerra italo-libica. La Tripolitania
divenne indipendente col re Idris I nel 1951. Nel 1969 Muammar Gheddafi
è arrivato al potere con un colpo di stato militare fondando la
Repubblica araba della Libia.

Il fazzoletto italiano con carta geografica del 1918 (vedi pag.57)

Dopo la fine della prima guerra mondiale in Italia nel 1918, comparve un
nuovo fazzoletto, sotto il titolo di ``ITALIA - REDENTA ED UNA PER
VALORE DEI SUOI SOLDATI - 3 NOVEMBRE 1918''. II 3 Novembre 1918 ha
segnato con la firma dell'armistizio tra l'Italia e Austria-Ungheria, la
fine della prima guerra mondiale su questo fronte. Questo fazzoletto
porta lungo i bordi il testo ``STAMPERIA ITALIANA DE ANGELI MILANO -
BREVETTATO''. Questo coincide con un fazzoletto di circa 52 x 52 cm
leggermente più piccolo rispetto al suo predecessore. Le colonie non
sono più mostrate, ma è stato eseguito un prolungamento a nord con i
territori ceduti dall'Austria-Ungheria, l'Alto Adige e l'Istria. Ciò ha
aumentato il numero di stemmi in circolo in modo tale che doveva essere
aggiunti al fondo di un'altra linea. Ciò comprende i distretti e le
nuove città di Ala, Arco, Bolzano, Capodistria, Gorizia, Istria (con
errori di stampa), Levico, Pola, Fiume, Parenzo, Pisino, Riva, Rovigno,
Rovereto, Volosca e Zara. Sul fazzoletto del 1918 si è rinunciato alla
popolazione e alle strutture militari.

I rapporti tra Francia e Italia

I tessuti italiani, in relazione con quelli francesi e tedeschi hanno
una maggiore quantità di colore, e ciò significa che gli stemmi sono
rappresentazioni su uno sfondo colorato. Per il fazzoletto per il fucile
nel 1891 così come per quelli pubblicati dopo la fine del secolo, c'è
una tiratura più recente di un colore solido di base con un colore
giallo ocra scuro al posto dei rosso. La questione dell'incisore dei
rulli della stampa per i fazzoletti italiani secondo lo studio dei
documenti esistenti fa riferimento alla famiglia francese Buquet. Qui si
possono trovare stampe successive di dettagli della stoffa del campione,
come ad esempio l'angolo in basso a sinistra dello stemma della lupa
romana con la ghirlanda,1 (vedi a pag. 62) e anche alcuni riferimenti
nel diario del 1887: ``2 Gennaio: \ldots{} viaggio a Rouen alla
stamperia Tervort, io gli mostro il primo testo del fazzoletto italiano
per il fucile 1870-1887\ldots{}'' ``12 Gennaio:\ldots{} Ho a che fare
con le iscrizioni del tessuto italiano..'' ``26 Gennaio: \ldots{}
Ottengo un messaggio di risposta che contiene un cilindro, inviato dal
signor De Angeli a Milano, per rinnovare la carta d'Italia, che ho
inciso quattro anni fa \ldots{} ,, 30 Gennaio:\ldots{} Trasferisco
l'emblema centrale del fazzoletto italiano \ldots{}'' Buquet aveva così
inciso nel 1883 per conto di De Angeli, il rullo di pressione per la
carta geografica del primo fazzoletto italiano e anni dopo, ha rinnovato
l'incisione. Inoltre è stato consegnato il fazzoletto delle istruzioni n
° 2. Qui il poco tempo trascorso è sorprendente, perché nel 1887 è stato
preparato il fazzoletto di istruzioni in parallelo con l'adozione
formale della carabina raffigurata in esso. Richiama in confronto il
fazzoletto francese n ° 1, che sembrava così in ritardo, che c'era già
un successore alla porta. Così si spiegherebbe in termini di fazzoletti
d'istruzione italiani, che le persone che si occupavano di incisione e
stampa erano responsabili. Tuttavia, rimane sconosciuto chi abbia
contribuito al contenuto militare. Per i paralleli con altri paesi,
dobbiamo dare per scontato che si trattasse di un ufficiale, forse anche
più d'uno.

\begin{verbatim}
                                [pic]

    Dettaglio dell’angolo in basso a sinistra del fazzoletto n°1

                           (veder pag. 40)

                                [pic]


                            15 maggio 1884
\end{verbatim}

Invio al ministero competente della descrizione dettagliata dei
contenuti del fazzoletto n°1

\begin{verbatim}
                                [pic]


                            6 giugno 1884
  Richiesta da parte di Giuseppe Frua al ministero competente, per
\end{verbatim}

autorizzazione alla stampa dei fazzoletti per istruzione militare
italiana

Capitolo III

\begin{verbatim}
                            I fazzoletti
                       per istruzione militare
                              in Italia

                     Produzioni della stamperia
                           De Angeli Frua
                              1884-1918


                   Seconda parte - Approfondimenti
\end{verbatim}

Il testo è stato ricavato dagli articoli di Dirk Ziesing e da alcune
scritte e significativi particolari dei singoli manufatti.

De Angeli-Frua

La stamperia italiana per i fazzoletti da istruzione è nata a Saronno,
cittadina a nord di Milano, fondata da Ernesto De Angeli (1849-1907).
Egli acquisì la propria formazione professionale grazie al barone
Costanzo Cantoni, fondatore dell'industria tessile in Italia. La sua
azienda di filatura e tessitura del cotone (cotonificio) nasceva a
Castellanza già dal 1845. Nel 1872 venne fondata la ``Società Ernesto De
Angeli \& C'' e nel 1878 denominata ``Stamperia milanese per stoffe
Cantoni''. De Angeli conquistò una grande reputazione nella sua città
natale e nel 1895 venne nominato senatore.

In questa situazione si deve però citare anche Giuseppe Frua (1855-1937)
che, già dall'età di 16 anni, cominciò a conoscere il lavoro della
tessitura presso un'azienda tedesca. Al suo rientro in Italia, dapprima
trovò un'occupazione nell'azienda tessile Caprotti quindi, nel 1875,
entrò a servizio di Eugenio Cantoni, figlio e successore del già
nominato Barone. Nel 1879 l'attività Frua venne spostata nella sede di
Castellanza. Nel 1883 egli sposò Anna De Angeli e nello stesso tempo
divenne procuratore dell'azienda del cognato. Dopodiché, nel 1890
divenne partner della fabbrica tessile dei fratelli Banfi di Legnano
(Anonima Frua \& Banfi) e nel 1896 si associano Ernesto De Angeli e
Giuseppe Frua concentrando filatura, tessitura e stampa in un'unica
azienda, denominata ``De Angli-Frua''.

All'inizio del ventesimo secolo (1900) si sperimentò anche la produzione
in proprio della seta. A tale scopo venne acquisto un terreno sul lago
di Orta adattando la precedente piantagione di viti e alberi da frutta
in gelsi, le cui foglie erano essenziali per l'allevamento dei bruchi.
Nello stesso periodo venne inoltre acquistato un sontuoso edificio del
18° secolo sul lago Maggiore, come sede di rappresentanza. Questa villa
De Angeli-Frua( a Laveno) -- come gran parte degli edifici di origine
aziendale a Milano e nei suoi dintorni -- esiste ancora oggi. In
definitiva, nel 1937 figuravano complessivamente cinque fabbriche con
11.000 dipendenti. La fabbrica di Gerenzano, vicino a Milano, divenne
oggetto di modello, così come la residenza del direttore e i vicini
edifici aziendali, cantine, scuole e un asilo aziendale. A seguito però
dei problemi economici dell'industria tessile europea derivante dalla
concorrenza asiatica, l'azienda chiuse le porte nel 1970..

Sguardo d'insieme

La stamperia tessile milanese si dedicò tra l'altro anche ai fazzoletti
militari d'istruzione. Per questo, attribuì particolare significato al
materiale cartografico, e in seguito venne integrata negli schemi di
numerazione italiani. Così nacque nel 1884 il primo fazzoletto con la
stampa di una scena militare italiana. Seguirono fazzoletti orientati
alla tecniche militari e alla fine vennero nuovamente rappresentati temi
geografici.

Fazzoletto italiano n° 1

La mappa sul fazzoletto in formato 64x58 cm mostra lo stivale italiano,
le isole maggiori di Sardegna e Sicilia così come le isole restanti.
Oltre alle strade di ordine (classe) 1, 2 e 3 e alle linee ferroviarie,
figurano i numeri dei dodici corpi d'armata e le sedi dei comandi e
delle fortezze.

Il titolo sovrimpresso recita ``Carta dimostrativa delle circoscrizioni
militari del regno d'Italia 1884''. Tutt'intorno sono rappresentati gli
stemmi delle diverse aree d'amministrazione italiane (come dire: le
provincie), con i numeri dei rispettivi abitanti. Negli angoli e al
centro dei lati figurano gli stemmi delle città maggiori: Roma, Firenze,
Napoli, Genova, Torino, Palermo, Milano e Venezia. Tra di essi appaiono
invece, in ordine alfabetico e in senso orario, tutte le altre città da
Alessandria a Vicenza.

Fazzoletto italiano n° 2

Il secondo fazzoletto della collezione italiana descrive il fucile della
fanteria Modello 1870-87. Esso riporta al centro la scritta ``Fucile
Modello 1870-87'' e sui margini le scritte ``Fazzoletto militare N° 2 --
(Privativa Industriale)'' nonché ``Stamperia E. De-Angeli \& C --
Milano''. La spiegazione delle due date consiste nel fatto che nel 1887
venne modernizzato il precedente modello del 1870. Un fazzoletto di
questa serie solo con il modello originale non venne alla luce, anche se
è certo che esistesse e che venne stampato dalla società Rolffs dedita
ai fazzoletti di istruzione austro-ungarici (vedi fazzoletto 2A di
pag.44 ). Con il loro aggiornamento, le precedenti armi vennero fornite
con il corredo di un caricatore più ampio. Il caricatore con magazzino
per quattro cartucce venne sviluppato per il capitano d'artiglieria
Giuseppe Vitali. Sullo stesso principio vennero del resto realizzati un
anno dopo anche i fucili olandesi Beaumont. Poiché le armi italiane
vennero adeguate in grandi quantità, è molto raro trovare i fucili
Vetterli proprio in versione originale. Per la guida del caricatore, la
sua carrozzeria riportava un'apertura nella parte inferiore. Per
compensare l'indebolimento della struttura, nel legno del fucile venne
integrato un piatto metallico. Oltre a ciò si integrò anche una barra
per la guida della punta della pallottola, mentre il coperchio di
protezione sull'apertura della cassa venne a sua volta rafforzata e
compensata con un anello metallico. Infine seguirono anche una modifica
della leva di sicurezza e l'irrobustimento della baionetta. Il mirino a
quadrante già abbozzato da Vecci rappresentò anch'esso una modifica
apportata sul Modello 1870 già nel 1881.

Fazzoletto italiano n° 3

I fucili costruiti nel 1887 vennero modificati già pochi anni dopo
attraverso un nuovo fucile a ripetizione. Anche per esso venne creato un
nuovo fazzoletto d'istruzione. Esso venne intitolato ``Fucile modello
1891'', con riporto sulla fascia esterna orizzontale del testo
``FAZZOLETTO MILITARE N°3 -- (PRIVATIVA INDUSTRIALE), nonché ``STAMPERIA
E. DE- ANGELI \& C -- MILANO''. Il nuovo fucile italiano venne
ufficialmente introdotto in data 29 marzo 1892. Esso è a struttura di
caricatore basato sul principio di Mannlicher. Ferdinando von Mannlicher
ne ricavò una licenza brevettuale di 300.000 lire. Il sistema di
otturazione venne progettato nell'arsenale cittadino dei fucili di
Torino sotto la direzione di Salvatore Carcano (1827-1903) e in
collaborazione con comandanti e ingegneri militari. In tale circostanza
viene talvolta fatto anche il nome di Gustavo Parravicino, Presidente
della Commissione per l'approvazione dei nuovi fucili. Dalla solida
costruzione, quest'arma aveva una capacità di magazzino di sei cartucce.
Il calibro relativamente piccolo di 6,5x52 mm determinò subito il
vantaggio che un singolo soldato poteva portare con sé un consistente
quantitativo di munizioni. D'altro lato la sua criticità generalmente
riconosciuta consisteva in operazioni di limitata portata balistica.
Anche perciò, le cartucce Carcano riportavano il proiettile con la punta
arrotondata, ancor oggi usata, e come già da tempo anche altre nazioni
avevano fatto con la punta dei loro proiettili. Un ulteriore svantaggio
proveniva dalla lunghezza del successivo modello realizzato per la
fanteria. Esso riporta un mirino scorrevole con un gittata da 450 a 2000
metri. La misura di 1,60 m dal calcio all'innesto della baionetta
derivava dal principio che, negli scontri con la cavalleria, un fante
potesse attaccare in modo efficace anche un cavaliere a cavallo. Per la
guerra di trincea, che si svilupperà durante la prima guerra mondiale,
questa lunghezza sarà invece di forte intralcio.

Perciò nella famiglia dei fucili Carcano comparvero delle successive
versioni accorciate. Il ``Moschetto Cavalleria'' venne sviluppato per la
Cavalleria nel 1893 e più tardi trasformato anche in fucile da caccia.

Tra le altre cose, è interessante la molteplicità delle varianti delle
cartucce, sempre stampate sul Fazzoletto Carcano, e riportate nella sua
parte bassa. Accanto alle normali cartucce sottili e pallottole con
copertura in nichel (``A pallottola''), c'erano quelle ``da
esercitazione'', quella ``a salve'' e le munizioni per tiro a segno. Una
particolarità consiste nelle pallottole a sparo singolo e in quelle a
mitraglia.

Fazzoletto cartografico italiano del 1912

L'aspetto del fazzoletto cartografico del 1912 riporta il medesimo
titolo della prima stampa del 1884 ed è identico anche il formato di
64x58 cm. La cornice degli stemmi rimane in sostanza la medesima, con le
seguenti modifiche: al posto di Porto-Maurizio subentra Imperia, dato
che era stata assorbita nella medesima zona rivierasca unitamente alla
vicina Oneglia. Rovigo viene collocata dopo Reggio Calabria e Reggio
Emilia, adesso in corretto ordine alfabetico. Inoltre, l'area sopra il
titolo viene riempita con cinque ulteriori stemmi. A quanto pare infatti
erano state semplicemente dimenticate le circoscrizioni di Campobasso,
Grosseto, Lecce, Potenza e Siracusa.

\begin{verbatim}
A  ciascuna  circoscrizione  viene  inoltre  associato  il  numero   dei
\end{verbatim}

residenti. Così, per esempio per Roma, figura l'indicazione di 864.851
abitanti. Viene inoltre riportato il numero complessivo degli italiani
nel 1912, per un valore di 34.978.634 unità con una densità di 268.682
persone a kmq. È anche presente un essenziale completamento alla carta
geografica con il riporto delle colonie possedute dall'Italia alla fine
del 19° secolo. Due dettagliate carte geografiche riportano il sovra
titolo ``POSSESSIONI D'OLTRE MARE COLONIA ERITREA -- SOMALIA -
TRIPOLITANIA'' (possedimenti d'oltre mare). Sul Mar Rosso era stata
acquisita la colonia italiana d'Eritrea nel 1890. Dopo la seconda guerra
mondiale questo territorio costituì una federazione con gli Etiopi. Solo
dopo una guerra di una decina d'anni l'Eritrea divenne uno stato
indipendente. Sul Corno d'Africa, al di sotto del Golfo di Aden, si
trovava la colonia ``Somalia Italiana''. Fazzoletto cartografico
italiano del 1918

Alla fine della prima guerra mondiale venne prodotto un altro fazzoletto
cartografico italiano, con titolo ``ITALIA -- REDENTA E UNA PER VALORE
DEI SUOI SOLDATI -- 3 NOVEMBRE 1918''. Il 3 novembre 1918 segnò
attraverso la firma dell'armistizio tra l'Italia e l'Austria-Ungheria la
fine della prima guerra mondiale su questo fronte. Questa Carta riporta
lungo la cornice la scritta ``STAMPERIA ITALIANA DE ANGELI MILANO --
BREVETTATO''. Questo fazzoletto con lati di circa 52x52 cm è leggermente
più piccolo del suo predecessore. Le aree coloniali non sono più
rappresentate, ma in cambio seguì un allargamento nel NORD attraverso i
territori de Sud Tirolo e del'Istria sottratti all'Austria-Ungheria.
Parimenti aumentò il numero degli stendardi di cornice, che venne
organizzato in una nuova fila disposta nell'area inferiore del
fazzoletto. Essa comprende le nuove circoscrizioni e sedi di Ala, Arco,
Bolzano, Capodistria, Gorizia, Istria (con un errore di stampa), Levico,
Pola, Fiume, Parenzo, Pisino, Riva, Rovigo, Rovereto, Volosca e Zara.
Nel fazzoletto del 1918 si soprassedette al riporto del numero degli
abitanti e delle strutture militari.

Nota esplicativa

Del fazzoletto sulla cavalleria (vedi pag. 52) l'unica cosa certa è che
il prodotto è della stamperia E. De Angeli. Attualmente non siamo in
grado di riportare altre documentazioni e/o spiegazioni. La stessa cosa
vale per il manufatto sulle armi pesanti che viene effigiato a pagina
53. Questi due fazzoletti però documentano che oltre ai fazzoletti
cartografici venivano stampati anche quelli dedicati a reparti militari
specifici.

\begin{verbatim}
                                [pic]

           Particolare fazzoletto militare italiano n° 2.
    Certificazione della produzione stampata sul bordo inferiore.
\end{verbatim}

{[}pic{]}

\begin{verbatim}
              Particolare del fazzoletto militare n° 1



                                [pic]



              Particolare del fazzoletto  militare n° 2
\end{verbatim}

{[}pic{]}

\begin{verbatim}
                        [pic]
\end{verbatim}

Particolari del fazzoletto militare n° 2. È evidente il nome del
produttore nel bordo inferiore del modello in alto

\begin{verbatim}
                                [pic]

             Particolare del fazzoletto militare n°2°A.

                                [pic]



          Altri particolari del fazzoletto militare  n° 2A

                                [pic]

                                [pic]

                                [pic]







                                [pic]



              Particolare del fazzoletto militare n° 3
                                [pic]

                                [pic]

           Altri particolari del fazzoletto militare n° 3
\end{verbatim}

{[}pic{]}

\begin{verbatim}
                                                                   [pic]


           Altri particolari del fazzoletto militare n° 3



                                [pic]


              Particolare del fazzoletto militare n° 4
\end{verbatim}

{[}pic{]}

\begin{verbatim}
                                                                   [pic]

              Particolari del fazzoletto militare n° 4
\end{verbatim}

{[}pic{]}{[}pic{]}

\begin{verbatim}
          Dettagli d’immagini del fazzoletto militare n° 4
                                [pic]
                                [pic]
                                [pic]
                                [pic]
                                [pic]

         Cartolina emessa per l’Arma della Regia Cavalleria
                          Periodo 1920-1930
                                [pic]

         Cartolina emessa per l’Arma della Regia Cavalleria
                          Periodo 1920-1930

                                [pic]

              Particolare del fazzoletto militare n° 5.
              Stemma di Milano e numero degli abitanti
                      della Provincia nel 1912
     Altre produzioni dell’epoca 1915-18 circa

I due seguenti fazzoletti vennero  prodotti  probabilmente  durante   il
\end{verbatim}

periodo della prima guerra mondiale. Il primo si riferisce all'igiene in
ambito domestico. Il secondo è dedicato all'educazione civica. Le frasi
riportate sono estrapolate da discorsi che Giuseppe Frua fece alle
maestranze in occasione di incontri per ricorrenze o conferimento di
attestati.

\begin{verbatim}
                                [pic]

                Fazzoletto di educazione famigliare.
                       Dimensioni: 25 x 25 cm



                                [pic]

   Fazzoletto di educazione civica  con scritte di Giuseppe Frua.
                       Dimensioni: 27 x 27 cm





                                [pic]

                    Foulard Etiopia del 1935/36 .
                       Dimensioni: 78 x 75 cm
\end{verbatim}

Probabilmente questa produzione fu fatta per il contributo dato
dall'Aeronautica militare italiana alla conquista di quel territorio.

\begin{verbatim}
                                [pic]

       Produzione casalinga di tessuti fine “800 e inizi “900
\end{verbatim}

Capitolo IV

\begin{verbatim}
                            Trame, stampe

                              e disegni

                              1930-1943
\end{verbatim}

Questo capitolo raccoglie alcune significative immagini dei principali
manufatti d'epoca pre-bellica e bellica del secondo conflitto mondiale.

{[}pic{]}

\begin{verbatim}
                                [pic]
\end{verbatim}

{[}pic{]}

\begin{verbatim}
                                [pic]
\end{verbatim}

{[}pic{]}

\begin{verbatim}
                                [pic]

                                                                   [pic]
\end{verbatim}

{[}pic{]}

\begin{verbatim}
                                [pic]


                                                                   [pic]
\end{verbatim}

{[}pic{]}

\begin{verbatim}
                                [pic]

                                                                   [pic]
\end{verbatim}

{[}pic{]}

\begin{verbatim}
                                [pic]

                                                                   [pic]

                                [pic]



                                [pic]
\end{verbatim}

Capitolo V

\begin{verbatim}
                         Tessuti artificiali
                              1938-1943
\end{verbatim}

{[}pic{]}

\begin{verbatim}
        Copertina della rivista trimestrale “I TESSILI NUOVI”
                    n° 28  Luglio-Settembre 1941
\end{verbatim}

La DAF negli anni '30 del XX° secolo era all'apice del livello
produttivo sia come qualità che quantità. Ma nel 1935 con la guerra
d'Etiopia l'Italia si trovò a dover contrastare l'embargo delle materie
prime voluto dalla Società delle Nazioni per iniziativa di Gran Bretagna
e Francia. Nel campo tessile per far fronte alla penuria di materiale
naturale vennero incrementate al massimo la ricerca e la produzione di
fibre sintetiche . Venivano così prodotti dei surrogati denominati
Lanital, Terital, Rayon ecc\ldots{} ecc\ldots{}. Tra le aziende
produttrici di tessuti con questi materiali imposti dalla autarchia vi
era la Snia Viscosa. Questa industria pubblicava una rivista trimestrale
che informava sulle iniziative aziendali e le attività produttive. Il
titolo della pubblicazione era ``I TESSILI NUOVI''. Dal numero 28 del
Luglio - Settembre 1941 e dal numero 31 dell'Aprile -- Giugno 1942 di
detto periodico apprendiamo che anche la DAF utilizzava tessuti SNIA per
i propri stampati. Sempre su questi numeri viene riferito che la grande
e milanesissima Sarta (a quei tempi si nominava così chi produceva alta
moda) Biki utilizzava prodotti SNIA per le proprie creazioni. Accostando
questi riferimenti è lecito supporre, anche se tuttora una conferma non
l'abbiamo, che Biki per i suoi lavori usasse prodotti DAF con materiali
sia sintetici che naturali. {[}pic{]}

Dalla rivista ``I TESSILI NUOVI'' n° 28 -- 1941 pag. 8

\begin{verbatim}
                                [pic]
   Dalla rivista “I TESSILI NUOVI”n° 31 Aprile-Giugno 1942 pag. 4
                A piè di foto la scritta BIKI MILANO

                                [pic]

         Dalla Rivista “I TESSILI NUOVI”n° 28- 1941 pag. 11

                                [pic]
                         Logo  SNIA VISCOSA

                                [pic]

                                [pic]

                                [pic]
                                [pic]
\end{verbatim}

{[}pic{]}

Capitolo VI

\begin{verbatim}
                        Tovaglie e tovaglioli
                              1950-1965
\end{verbatim}

Immagini di prodotti dal largo impiego in ambito domestico e familiare.

\begin{verbatim}
                                [pic]

                              Tovaglia
                      Dimensioni: cm 134 x 124
                           -particolare-.


                                [pic]
             La stessa tovaglia della pagina precedente
                     con dettagli fronte e retro
                                [pic]

                              Tovaglia.
                      Dimensioni: 250 x 136 cm
                            -particolare-
                                [pic]


       Pezza di prova per produzione di tovaglie e tovaglioli


                                [pic]
                   Tovaglietta da prima colazione.
                       Dimensioni: 44 x 33 cm


                                [pic]
                         Grembiule da cucina
\end{verbatim}

Capitolo VII

\begin{verbatim}
                        Strofinacci da cucina
                              1965-1970
\end{verbatim}

Una raccolta di immagini dei più utili e indispensabili strumenti di
casa, spesso impreziositi da disegni pieni di buon gusto e allegria.
{[}pic{]}{[}pic{]} {[}pic{]}{[}pic{]} {[}pic{]}{[}pic{]} {[}pic{]}

\begin{verbatim}
                                [pic]


                                [pic]
\end{verbatim}

Capitolo VIII

\begin{verbatim}
                  Pubblicità promozione e immagine




  La DAF ha propagandato la propria  produzione  in  svariate  forme  ma
\end{verbatim}

soprattutto sui periodici di moda femminile: Vesta, Vendere, Mani di
Fata. I prodotti lavorati erano cotone, satin, taffettà, materassè, seta
da paracadute, seta pura italiana, georgette. Il marchio di fabbrica era
``Sole Onda'', come riportato in figura.

\begin{verbatim}
                                [pic]
                    Marchio di fabbrica della DAF
\end{verbatim}

I nomi originali dei principali prodotti erano: Telene (tela), Sol
(cretonnè), Costella (tessuto per abiti), Velita (voilè a doppio ritorto
per abiti), Radiosa (seta artificiale per abiti), Tuxo (rayon, fibra
artificiale), Retex (seta e lana).

\begin{verbatim}
                                [pic]

                                [pic]


             Da ”Mani di Fata” del dicembre 1932  n° 12


                                [pic]

              Pagina pubblicitaria per tessuti stampati


                                [pic]

                    Pagina pubblicitaria del 1938
                                [pic]

            Da “Mani di Fata” dell’ agosto 1931, pag. 19
                                [pic]

   Da “L’illustrazione italiana” n° 22 del  giugno 1930, pag. 978

                                [pic]

             Da “Mani di Fata” del luglio 1932, pag. 15

                                [pic]

             Da “Mani di Fata” dell’agosto 1932, pag. 19
\end{verbatim}

{[}pic{]}

\begin{verbatim}
                                [pic]

                        Locandina tranviaria
\end{verbatim}

{[}pic{]} Produzione di fazzoletti con personaggi Walt Disney

\begin{verbatim}
                                                                   [pic]


                                  Cartolina a firma Boccasile
\end{verbatim}

{[}pic{]}{[}pic{]}

Serie di sei cartoline pubblicitarie emesse nel 1939

{[}pic{]}{[}pic{]} {[}pic{]}{[}pic{]} {[}pic{]}

\begin{verbatim}
     Stampigliatura sul retro delle sei  cartoline pubblicitarie
\end{verbatim}

{[}pic{]}{[}pic{]} {[}pic{]}

\begin{verbatim}
       Dal periodico della Snia Viscosa 1938-1942. Numeri vari
                             RIFERIMENTI
\end{verbatim}

Ideatore dell'opera: Loredano Tavazzi

Elaborazione immagini e inserimento testi: Giancarlo Soave Niccolò
Zucchi Frua Impostazione, coordinamento, ricerca, scelta immagini e
didascalie: Loredano Tavazzi

\begin{verbatim}
                                  ●
\end{verbatim}

I documenti per i fazzoletti di istruzione militare provengono
dall'archivio del Dott. Dirk Ziesing

Tutti gli altri documenti provengono dall'archivio Loredano Tavazzi

\begin{verbatim}
                                  ●
\end{verbatim}

Per le traduzioni da tedesco, francese e inglese hanno contribuito:

Alessandro Porro, docente dell'Università di Brescia

Cinzia Pozzi. interprete

Carlo Solarino , giornalista

\begin{verbatim}
                          AUTORI DEI TESTI

                           - Marina Frua -

                              Premessa



                                  ●
                           -Dirk Ziesing-

      Military instruction hanndkerchief of the British Empire

                       Italien- De Angeli Frua



                                  ●

                         -Loredano Tavazzi -

   Estrapolazioni dai testi originali di Dirk Ziesing e di alcuni
          significativi particolari dai singoli manufatti.

      Note esplicative su fazzoletti cavalleria e armi pesanti.

              Note esplicative su fazzoletti 1915–1918.

                Nota esplicativa sul foulard Etiopia.

              Nota esplicativa sui tessuti artificiali.

            Note esplicative su pubblicità e promozione.





                    INDICE DEI TESTI E ARGOMENTI
\end{verbatim}

Premessa pag. 5

Capitolo I

• La nascita dei fazzoletti per istruzione militare pag. 7

• Military instruction handkerchief of the British Empire pag. 8 • I
fazzoletti con istruzioni militari dell'Impero Britannico pag. 9

• Il primo fazzoletto britannico con istruzioni pag. 11 • Il secondo
fazzoletto britannico con istruzioni pag. 17 • Il terzo fazzoletto
britannico con istruzioni pag. 25 • L'origine dei fazzoletti con
istruzioni militari pag. 31

Capitolo II

\begin{verbatim}
                                            • I fazzoletti per
                                              istruzione militare in
                                              Italia
\end{verbatim}

prima parte: descrizione storica pag. 37

• De Angeli Frua pag. 39 • Sguardo d'insieme pag. 39 • Il fazzoletto
italiano n°1 pag. 39 • Il fazzoletto italiano n°2 pag. 43 • Il
fazzoletto italiano n°3 pag. 43 • Il fazzoletto italiano con carta
geografica del 1912 pag. 59 • Il fazzoletto italiano con carta
geografica del 1918 pag. 61 • I rapporti tra Francia e Italia pag. 61

Capitolo III

\begin{verbatim}
          • I fazzoletti per istruzione militare in Italia - Seconda
            parte
                                               pag. 65

            pag.   65
     • De Angeli Frua                                       pag. 66
     • Sguardo d’insieme                                        pag.  67
     • Il fazzoletto italiano n° 1
         pag.  68
     • Il fazzoletto italiano n° 2
         pag.  68
     • Il fazzoletto italiano n° 3
         pag.  69
     • Fazzoletto cartografico italiano del 1912                pag.  70
     • Fazzoletto cartografico italiano del 1918
       pag. 72
     • Nota esplicativa
              pag.  72
     • Altre produzioni dell’epoca
       pag.  95
     • Specifica su fazzoletto Etiopia 1935-36                  pag.  97
\end{verbatim}

Capitolo IV • Tessuti disegni e trame del periodo 1930-1943 pag. 99
Capitolo V • Tessuti artificiali del periodo 1938-1943 pag. 109 Capitolo
VI • Tovaglie, tovaglioli e altro del periodo 1950-1965 pag. 119
Capitolo VII • Strofinacci da cucina del periodo 1965-1970 pag. 127

Capitolo VIII • Pubblicità, promozioni, immagine pag. 137 • Specifica
sulla pubblicità pag. 138

• Riferimenti pag. 155 • Autori dei testi pag. 156 INDICE DELLE
ILLUSTRAZIONI

Ciclo di lavorazione della seta, anno 1933 pag. 2 Prima pagina della
rivista inglese sulla nascita dei fazzoletti di istruzione militare pag.
12

Fazzoletto militare britannico n°1 pag. 13 Particolari del fazzoletto
n°1 pag. 14-15 Fazzoletto militare britannico n°2 pag. 18 Particolare
del fazzoletto n°2 pag. 19 Fazzoletto militare britannico della marina
n°3 pag. 22 Fazzoletto britannico della Marina -particolare pag. 23
Fazzoletto militare italiano n°1 pag. 40 Fazzoletto militare italiano
n°2 pag. 41 Particolare del fazzoletto militare italiano n°2A pag. 44
Fazzoletto militare italiano n°3 pag. 45 Fazzoletto militare italiano
n°4 --Cavalleria- pag. 52 Fazzoletto delle armi pesanti pag. 53
Fazzoletto militare italiano n°5 cartografico 1912 pag. 56 Fazzoletto
militare italiano n°6 cartografico 1918 pag. 57 Dettaglio in basso a
sinistra del fazzoletto n°1 pag. 62 Documento descrizione contenuti
fazzoletto n°1 pag. 63 Richiesta di G. Frua per stampa fazzoletti
militari pag. 64 Certificazione della produzione De Angeli pag. 73
Particolari del fazzoletto militare italiano n°2 pag. 73-75-76
Particolare del fazzoletto militare italiano n°1 pag. 74 Particolari del
fazzoletto militare italiano n°2A pag.77 a 81 Fazzoletto militare
italiano n°4 per Cavalleria pag. 80 Particolari del fazzoletto militare
italiano n°4 pag. 81-83

Particolare del fazzoletto militare italiano n°3 pag. da 82 a 84
Particolari del fazzoletto militare italiano n°4 pag. 88-91 Cartoline
della Regia Cavalleria 1920-1930 pag. 92-93 Particolare del fazzoletto
militare italiano n°5 pag. 94 Fazzoletto di educazione famigliare pag.
95 Fazzoletto di educazione civica pag. 96 Foulard della guerra
d'Etiopia 1935-1936 pag. 97 Produzione tessile di fine `800 pag. 98
Disegni di tessuti pag. 100-108 ``I tessili nuovi'' 1941, copertina pag.
110 ``I tessili nuovi'' 1941, pagine interne pag. 112-114 Logo Snia
Viscosa pag. 114 La stilista Biki (da Corsera) pag. 115 Ciclo di
produzione del nylon pag. 116 Ciclo di produzione del nàion e del fiocco
pag. 117 Particolari tovaglie pag. 120-122 Prove produzione tovaglie e
tovaglioli pag. 123 Tovaglietta per prima colazione pag. 124 Grembiule
da cucina pag. 125 Strofinacci da cucina pag. 128- 136 Intestazione
``Mani di fata'' pag. 139 Intestazione ``L'illustrazione italiana'' pag.
139 ``Mani di fata'', pagine interne pag. 140-142 Locandina tramviaria
pag. 143 Fazzoletti con personaggi W. Disney pag. 144 Cartoline pag.
144-148 Pagine e immagini pubblicitarie pag. 149-154
----------------------- \textasciitilde{}::
